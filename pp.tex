\documentclass[11pt]{beamer}
\usetheme{Madrid}
\usefonttheme{serif}

\usepackage[utf8]{inputenc}
\usepackage[T1]{fontenc}

\usepackage{amsmath}
\usepackage{amsfonts}
\usepackage{amssymb}
\usepackage{graphicx}
\usepackage{listings}
\usepackage{float}
\usepackage{subcaption}
\usepackage{forest}

\lstset{
    language=Python,
    basicstyle=\ttfamily\footnotesize,
    keywordstyle=\color{blue},
    commentstyle=\color{gray},
    stringstyle=\color{orange},
    showstringspaces=false,
    frame=single,
    numbers=left,
    numberstyle=\tiny,
    numbersep=5pt,
    breaklines=true,
    breakatwhitespace=true,
    basicstyle=\ttfamily\scriptsize,
    columns=fullflexible,
    inputencoding=utf8,
    literate=
      {é}{{\'e}}1 {è}{{\`e}}1 {ê}{{\^e}}1 {ë}{{\"e}}1
      {à}{{\`a}}1 {â}{{\^a}}1 {ä}{{\"a}}1
      {ù}{{\`u}}1 {û}{{\^u}}1 {ü}{{\"u}}1
      {ô}{{\^o}}1 {ö}{{\"o}}1 {î}{{\^i}}1 {ï}{{\"i}}1
      {ç}{{\c{c}}}1
      {É}{{\'E}}1 {È}{{\`E}}1 {Ê}{{\^E}}1 {Ë}{{\"E}}1
}

\setbeamertemplate{caption}[numbered]

\author[RULLAC FAUCON]{Auteurs \\ RULLAC Éloïse, FAUCON Léo}
\title{UE622 - Génie Logiciel}
\date{May 5, 2025} 

\bibliographystyle{apalike}

\begin{document}

\begin{frame}
\titlepage
\end{frame}

\begin{frame}{Répartition des tâches}
    \centering
    \begin{itemize}
        \item \textbf{En groupe} : Cahier des charges, axe fonctionnel, statique, dynamique.
        \item \textbf{Éloïse} : IHM, UI, controleurs (front-end).
        \item \textbf{Léo} : Communication client-serveur (back-end).
    \end{itemize}
\end{frame}

\begin{frame}{Graphique de Gantt}
    \begin{figure}[H]
        \centering
            \includegraphics[width=\linewidth]{captures/11.png}
            \caption{Graphique de Gantt}
    \end{figure}
\end{frame}

\begin{frame}{Exemple d'exigences}
    \centering
    \begin{itemize}
        \item AU010 : Le système doit permettre à l'utilisateur de se connecter via une adresse mail et un mot de passe.
        \item AU020 : Le système doit permettre à l'utilisateur de gérer son compte (adresse mail, mot de passe, nom, prénom).
    \end{itemize}
\end{frame}

\begin{frame}
    \begin{figure}[H]
        \centering
            \includegraphics[width=\linewidth]{captures/9.png}
            \caption{Cas d'usages (1/2)}
    \end{figure}
\end{frame}

\begin{frame}
    \begin{figure}[H]
        \centering
            \includegraphics[width=\linewidth]{captures/10.png}
            \caption{Cas d'usages (2/2)}
    \end{figure}
\end{frame}

\begin{frame}
\begin{figure}[H]
    \centering
    \begin{subfigure}{0.45\textwidth}
        \includegraphics[width=\linewidth]{captures/5.png}
        \caption{Maquettage}
    \end{subfigure}
    \vfill
    \begin{subfigure}{0.45\textwidth}
        \includegraphics[width=\linewidth]{captures/1.png}
        \caption{home\_page.py}
    \end{subfigure}
\end{figure}
\end{frame}
\begin{frame}
\begin{figure}[H]
    \centering
    \begin{subfigure}{0.45\textwidth}
        \includegraphics[width=\linewidth]{captures/6.png}
        \caption{Maquettage}
    \end{subfigure}
    \vfill
    \begin{subfigure}{0.45\textwidth}
        \includegraphics[width=\linewidth]{captures/2.png}
        \caption{login\_page.py}
    \end{subfigure}
\end{figure}
\end{frame}
\begin{frame}
\begin{figure}[H]
    \centering
    \begin{subfigure}{0.45\textwidth}
        \includegraphics[width=\linewidth]{captures/7.png}
        \caption{Maquettage}
    \end{subfigure}
    \vfill
    \begin{subfigure}{0.45\textwidth}
        \includegraphics[width=\linewidth]{captures/3.png}
        \caption{sign\_in\_page.py}
    \end{subfigure}
\end{figure}
\end{frame}
\begin{frame}
\begin{figure}[H]
    \centering
    \begin{subfigure}{0.45\textwidth}
        \includegraphics[width=\linewidth]{captures/8.png}
        \caption{Maquettage}
    \end{subfigure}
    \vfill
    \begin{subfigure}{0.45\textwidth}
        \includegraphics[width=\linewidth]{captures/4.png}
        \caption{event\_page.py}
    \end{subfigure}
\end{figure}
\end{frame}

\begin{frame}{Arborescence du projet}
    \centering
    \begin{forest}
    [src
        [DAO]
        [dataclass]
        [factory]
        [menus 
            [diairies]
            [event]
            [task]
        ]
        [pages]
        [ui]
    ]
    \end{forest}
\end{frame}

\begin{frame}{Communication client-serveur - Modules}
    \begin{itemize}
        \item Communucation client-serveur: socket.
        \item Communication serveur-BDD : sqlite3.    
    \end{itemize}
\end{frame}

\begin{frame}{CCS - Opérations}
    \centering
    \begin{itemize}
        \item \textbf{1}: création de compte
        \item \textbf{2}: connexion à un compte
        \item \textbf{3}: requête du client avec comme spécification la catégorie $requestType$
        \item \textbf{4}: retour serveur positif
        \item \textbf{5}: retour serveur négatif avec message d'erreur dans la catégorie $errmsg$
    \end{itemize}
\end{frame}

\begin{frame}{CCS - RequestType}
    \begin{itemize}
        \item \textbf{"createAgenda"}: création d'un agenda
        \item \textbf{"updateAgenda"}: mise à jour d'un agenda
        \item \textbf{"getAgendaList"}: récupération de la liste des agendas pour un utilisateur
        \item \textbf{"deleteAgenda"}: suppression d'un agenda
        \item ...
    \end{itemize}
\end{frame}

\begin{lstlisting}
class Request:
    def __init__(self, query: str, func: Callable[[sqlite3.Cursor], str]):
        self.query = query
        self.func = func

    def send(self, *args) -> bytes:
        return self.func(con.execute(self.query, args)).encode()

def createAgenda(cur: sqlite3.Cursor) -> str:
    return "{\"data\":{\"agenda_id\":" + str(cur.lastrowid) + "},\"op\":4}"

requestType_to_query: dict[str, Request] = {
    "createAgenda": Request(
        "insert into agenda (user_id, name) values (?, ?);",
        createAgenda
    )
}

# if m_json["op"] == 3:
try:
    data: bytes = requestType_to_query[m_json["requestType"]].send(*m_json["data"].values())
except sqlite3.Error as e:
    conn.sendall(("{\"errmsg\":\"" + str(e) + "\",\"op\":5}").encode())
else:
    conn.sendall(data)
\end{lstlisting}

\newpage
\begin{lstlisting}
class AgendaDAO:
    def delete(self, agenda: Agenda):
        self.dbcom.sendall({
            "data":{
                "agenda_id": agenda.id
            },
            "requestType": "deleteAgenda",
            "op": 3
        })
        return self.dbcom.recv()
    
    def get_list(self, user: User) -> list[Agenda]:
        self.dbcom.sendall({
            "data": {
                "user_id": user.id
            },
            "requestType": "getAgendaList",
            "op": 3
        })
        r: list[Agenda] = []
        data: dict = self.dbcom.recv()
        for agenda in data["data"]["agendaList"]:
            r.append(Agenda(agenda["id"], agenda["name"]))

        return r
\end{lstlisting}

\begin{frame}
    \begin{figure}[H]
        \centering
            \includegraphics[width=\linewidth]{captures/12.png}
            \caption{MEA}
    \end{figure}
\end{frame}

\end{document}